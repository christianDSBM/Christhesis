\chapter{Digráficas de intervalos ajustadas}
\label{cap:DigrafIntAj}

En el año 1989 se publicó un articulo de M. Sen, S. Das, A.B. Roy y D.B. West en el  \textit{"Journal of graph theory"}. En este introduce una nueva definición que generalizaría a las gráficas de intervalos. Dicha generalización iría enfocada a tener caracterizaciones análogas a las de las gráficas de intervalos, una de ellas, la propiedad de los unos consecutivos para columnas de la matriz asociada a la gráfica de clanes maximales. 
Desafortunadamente esta definición, no contaría con un alguna caracterización en términos de estructuras prohibidas, así como se tiene para las gráficas de intervalos, las cuales hemos visto que tienen ciertas estructuras prohibidas como las tripletas asteroidales o cuatro ciclos sin cuerdas. 
Así, al solo contar con la caracterización dada por la propiedad de los unos consecutivos, se tendría un algoritmo de reconocimiento de tiempo polinomial.

Ellos definen un digráfica de intervalos como una digráfica $G$ la cual admite una representación de par de intervalos. Es decir, para cada vértice $v\in V(G)$, existen $I_v,J_v$ un par de intervalos, y $uv\in E(G)$ si y solo si $I_u$ intersecta a $J_v$. 

Veamos nuevamente varios ejemplos a fin de familiarizarnos.

r