\chapter*{Prefacio}
\label{sec:prfcio}

Las gráficas de intervalos tienen un gran abanico de aplicaciones en el mundo real, pues son capaces de modelar problemas los cuales dependen de una variable en común. Hay muchos ejemplos de esto, tal vez el mas sencillo sea el de acomodar cátedras en un auditorio, en un horario de servicio dado. De forma que se puedan llevar a cabo el mayor número de cátedras y que no haya empalmes entre estas. Aquí las tareas a realizarse son las de dar las cátedras y estas dependen del tiempo que tomará cada una de estas, esto se puede modelar asiganando a cada catedra un vértice y si sucede que los horarios de dos clases se empalman, diremos que sus vértices correspondientes son \textit{adyacentes}. Otro ejemplo un poco menos evidente es el de la modelación de luces de un semáforo, en donde ahora se hace uso de un carril para la circulación vehicular sin incidentes. Es decir hay que programar eficientemente el semáforo de tal forma que no haya flujo vehicular incompatible, evitando así accidentes. Otro problema modelable es el de uso de memoria en una computadora, para guardar (reusar) nombres de variables, aquí nuevamente a cada variable le corresponde un vértice y estos serán adyacentes si los nombres se ocupan al mismo tiempo.

Además de resultar tan útiles en el mundo de las matemáticas aplicadas, estas tienen propiedades interesantes, es decir son una familia de gráficas muy ricas en cuanto a estructura se refiere, por ejemplo gracias a sus caracterizaciones se cuenta con un algoritmo de reconocimiento de tiempo lineal, es decir dada una gráfica se puede comprobar si esta es o no gráfica de intervalos. 

Naturalmente, nos gustaría contar con un concepto que extienda el concepto de gráficas de intervalos a gráficas dirigidas.
En realidad ya se cuenta con una definición para gráficas dirigidas, la cual posee una relación de incidencia similar a la dada para las gráficas de intervalos, las digráficas de intervalos, estas, a diferencia del caso no dirigido, no cuentan de momento con un algoritmo de reconocimiento lineal. Si se cuenta con un algoritmo de tiempo polinomial, sin embargo para gráficas incluso no tan grandes, los tiempos de reconocimiento son significativamente altos. 
Estas admiten una caracterización en términos de su matriz de incidencia pero nuevamente a diferencia del caso no dirigido, no contamos con una caracterización de estructura.

En este punto, viendo todas las carencias de las digráficas de intervalos respecto al caso no dirigido, es donde se recoge la definición que introducen Tomás Feder, et al. Ellos dan una generalización de las gráficas de intervalos, las digráficas de intervalos ajustadas, con esta nueva definición, obtendremos una caracterización de estructura, en termino de una estructura noble que llamaremos \textit{el par invertible}. Con esta nueva caracterización daremos un algoritmo de reconocimiento de tiempo polinomial bajo, de complejidad $O(n^3)$. Otra cosa inmediata que se obtiene es el hecho que tenemos nuevamente la propiedad reflexiva(así como las gráficas de intervalos), de la cual carecen las digráficas de intervalos.

Así, centraremos nuestros esfuerzos en esta tesis en hacer un breve estudio sobre este tipo de digráficas.