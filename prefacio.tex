\chapter*{Prefacio}
\label{sec:prfcio}

Las gráficas de intervalos tienen un gran abanico de aplicaciones en el mundo real, esto tal vez de la necesidad de resolver una serie de problemas en función de una variable en común (muchas veces el tiempo) por ejemplo una gráfica de intervalos puede modelar problemas en el cual haya una serie de solicitudes para hacer uso de una instalación o un servicio, por ejemplo modelar las luces de un semáforo, en donde las luces del semáforo hacen peticiones para el uso del carril, de forma que no haya carros en sentidos incompatibles, o por ejemplo usos de aulas en una escuela en función de los horarios de clase, o uso de memoria en una computadora. 

Además de resultar tan útiles en el mundo de las matemáticas aplicadas, estas tienen propiedades interesantes, es decir son una familia de gráficas muy ricas en cuanto a estructura se refiere, por ejemplo gracias a sus caracterizaciones se cuenta con un algoritmo de reconocimiento de tiempo lineal, es decir dada una gráfica se puede comprobar siesta es o no gráfica de intervalos. 

Naturalmente, nos gustaría contar con un concepto que extienda el concepto de gráficas de intervalos a gráficas dirigidas. 
En realidad ya se cuenta con una definición para gráficas dirigidas, la cual posee una relación de incidencia similar a la dada para las gráficas de intervalos, las digráficas de intervalos, estas, a diferencia del caso no dirigido, no cuentan de momento con un algoritmo de reconocimiento lineal. Si se cuenta con un algoritmo de tiempo polinomial, sin embargo para gráficas incluso no tan grandes, los tiempos de reconocimiento son significativamente altos. 
Estas admiten una caracterización en términos de su matriz de incidencia pero nuevamente a diferencia del caso no dirigido, no contamos con una caracterización de estructura.

En este punto, viendo todas las carencias de las digráficas de intervalos respecto al caso no dirigido, es donde se recoge la definición que introducen Tomás Feder, et al. Ellos dan una generalización de las gráficas de intervalos las digraficas de intervalos ajustadas, con esta nueva definición, obtendremos una caracterización de estructura, en termino de una estructura noble que llamaremos "el par invertible". Con esta nueva caracterización daremos un algoritmo de reconocimiento de tiempo polinomial bajo, de complejidad $O(n^3)$. Otra cosa inmediata que se obtiene es el hecho que tenemos nuevamente la propiedad reflexiva(así como las gráficas de intervalos), de la cual se carecen las digráficas de intervalos.

Así, centraremos nuestros esfuerzos en esta tesis en hacer un breve estudio sobre este tipo de digráficas.