\chapter{Introducci\'on}
La teoría de gráficas es un campo de las matemáticas ampliamente estudiado
debido a la fascinante capacidad que tienen las gráficas para resolver problemas
del mundo real. Las gráficas son objetos matemáticos utilizados frecuentemente
como herramienta para representar y modelar diversas situaciones. Se dice que la
teoría de las gráficas se originó en el siglo XVIII, específicamente al modelar
el famoso problema de los puentes de Königsberg. Para contextualizar, la ciudad
de Königsberg (hoy Kaliningrado) estaba atravesada por el río Pregel, el cual
formaba una isla. Además, el río se bifurcaba, generando así cuatro regiones
distintas, como se muestra en la imagen. Estas regiones de tierra estaban unidas
por siete puentes, y el problema consistía en trazar una trayectoria que pasara
una sola vez por cada uno de los siete puentes, empezando y terminando en la
misma región de tierra. En 1735, el destacado matemático suizo Leonhard Euler
resolvió este problema al modelarlo mediante una representación gráfica. Euler
asignó un punto a cada región de tierra y utilizó líneas para representar los
siete puentes, creando así una gráfica. Al estudiar la estructura de esta
gráfica, Euler observó que había por los cuatro puntos había un número impar de
puentes. Utilizando un razonamiento que ahora se conoce como el teorema de
Euler, demostró que no era posible trazar una trayectoria que cruzara cada
puente una sola vez y que terminara en el punto inicial.

Esta solución de Euler al problema de los puentes de Königsberg sentó las bases
de la teoría de gráficas. Su enfoque para resolver el problema mediante una
representación gráfica y su demostración del teorema de Euler allanaron el
camino para el desarrollo de conceptos y resultados fundamentales en la teoría
de gráficas. Desde entonces, la teoría de gráficas ha encontrado aplicaciones en
una amplia gama de disciplinas, incluyendo la ciencia de la computación, la
logística, la biología, la física y la sociología. Por ejemplo, en el ámbito de
la logística, las gráficas se utilizan para determinar la ruta más eficiente
para el transporte de mercancías o para la planificación de redes de
distribución. Además, en la sociología, las gráficas son una herramienta valiosa
para analizar y comprender las interacciones sociales, identificar líderes de
opinión y comprender los patrones de difusión de información. En el campo de la
informática, la teoría de gráficas es fundamental en el diseño y análisis de
algoritmos. Los algoritmos de búsqueda, recorrido y optimización en gráficas son
ampliamente utilizados en la resolución de problemas computacionales, como la
planificación de rutas, el enrutamiento de redes y el análisis de datos en redes
complejas.

En resumen, modelar un problema del mundo real mediante una gráfica nos
proporciona una nueva perspectiva sobre el mismo. Podemos analizar la estructura
generada por la gráfica y descubrir información y patrones que no eran evidentes
inicialmente. Al reinterpretar esta información en el contexto del problema
real, podemos tomar decisiones informadas sobre cómo seguir atacando nuestro
problema. Además, al interpretar los resultados de vuelta al mundo real, es
posible realizar ajustes, mejoras o modificaciones en función de la comprensión
obtenida a través de la modelización de la gráfica. Estos cambios pueden llevar
a soluciones más eficientes, eficaces o innovadoras, y ayudar a resolver
problemas de manera más efectiva.


\section{Conceptos básicos}
\label{sec:Cncpts bscs}

En esta sección, nos enfocaremos en presentar las definiciones necesarias para
abordar el tema central de la tesis. Si bien esta sección puede resultar densa
debido a la cantidad de definiciones, nos esforzaremos por brindar numerosos
ejemplos de gráficas que faciliten la comprensión de estos nuevos conceptos.

En primer lugar, es crucial establecer un sólido fundamento conceptual para
adentrarnos en el estudio de las gráficas. Por lo tanto, se presentarán las
definiciones clave, empezando por la definición de gráfica así como las
definiciones de vértices, aristas, tamaño y orden de una gráfica, y otros
elementos esenciales para su comprensión.

Comencemos con el concepto mas fundamental, el de \indice{gráfica}. Una
\textbf{gr\'afica} \indiceSub{gráfica}{no dirigida} $G$, es un par ordenado
$(V(G), E(G))$ el cual consiste de un par de conjuntos, $V(G)$ un conjunto
finito cuyos elementos llamamos \indice{vértices} y $E(G)$ un conjunto cuyos
elementos llamamos \indice{aristas}, los cuales son parejas no ordenadas de
vértices, así $E(G)\subseteq \{ \{x,y\} \colon\ x,y\in V(G) \}$.  Si $G$ es una
gráfica y $(V(G),E(G))$ son sus conjuntos de vértices y aristas,
respectivamente, entonces diremos que, el \indice{orden} de $G$ es $|V(G)|$ y
que $|E(G)|$ es el \indice{tamaño} de $G$.

También tenemos la definición de \textbf{gr\'afica}
\indiceSub{gráfica}{dirigida}, donde el único cambio es que el conjunto de
aristas son ahora pares ordenados, es decir la definición es la siguiente. Una
gráfica dirigida $G=(V(G),F(G))$ es una pareja ordenada que consta de un
conjunto arbitrario de vértices $V(G)$ y de un conjunto de \indice{flechas}
$F(G)\subseteq \{ (x,y) | x,y\in V(G) \}$.   An\'alogamente al caso de las
gr\'aficas, es posible definir orden y tama\~no para una gr\'afica dirigida. Se
suele nombrar a las gráficas dirigdas por digráficas y esta será la forma como
nos referiremos a las gráficas dirigidas por el resto de la tesis. Por el
momento, dejaremos de lado esta definición  y la retomaremos en capítulos
posteriores.


Veamos a continuación los siguientes ejemplos. Comencemos con $V(G)=\{1,2,3\}$
$E(G)=\{ \{1,2\} \}$. Claramente, $G. = (V(G), E(G))$ es un ejemplo de una
gráfica de orden tres y de tamaño uno.   Notemos que, para expresar una
gr\'afica un poco m\'as grande, por ejemplo, de orden seis y tama\~no doce,
escribir a los conjuntos de v\'ertices y aristas se vuelve una labor tediosa.
Por lo anterior, es deseable contar con alguna forma m\'as eficiente de
representar a $G$

    
La teoría de gráficas tiene una gran ventaja respecto a otras ramas de las
matemáticas, en principio, según nuestra definición de gráfica, todo es
discreto, y como segunda cosa, la mayoría de nuestros objetos pueden ser
representados de forma gráfica. Por ello, haremos los siguiente.  A cada
elemento de $V(G)$ lo representamos por medio de un pequeño circulo, adem\'as,
cada uno tendrá una pequeña etiqueta que nos indicará de alguna forma
conveniente de qué elemento de $V(G)$ se trata. Las aristas son representadas
por medio de lineas, que empiezan y terminan en los círculos correspondientes a
su par de vértices. Las aristas, a diferencia de los vértices, no suelen ir
etiquetadas, sin embargo uno podría etiquetar a una arista con $uv$ o $vu$ si
esta arista une precisamente a los vértices $u$ y $v$. Esta será la forma como
se hará referencia a las aristas a lo largo de este texto. Notemos que toda
arista forzosamente se corresponde con dos vértices (no necesariamente
distintos), así toda arista, debe ser dibujada con sus extremos, inicial y
final, anclados en los vértices (los círculos).

\begin{figure}[H]
  \centering
  \includegraphics[width=0.4\textwidth]{recursos/capturas/01(1).jpg}
  \caption{Representación de una gráfica, de tamaño uno y orden dos.}
  \label{fig:01}
\end{figure}

Como se vio en el ejemplo anterior, puede haber algún vértice que no esté
``conectado'' a otro vértice, como el vértice 3. Decimos que dos vértices que
est\'an conectados (que pertenecen a la misma arista), como el 1 y 2, son
\indiceSub{vértice}{adyacente}\textbf{s}. En otro caso diremos que son
\indiceSub{vértice}{independiente}\textbf{s} o no adyacentes. Por ejemplo los
v\'ertices 1 y 3 son independientes, y tambi\'en los v\'ertices 2 y 3 lo son. En
general, decimos que un conjunto $S\subseteq V(G)$ es  un
\indiceSub{conjunto}{independiente} si y solo si, para cualesquiera v\' ertices
$x,y$, si $x$ es distinto de $y$, entonces $xy\notin E(G)$.

Notemos que si $E(G)$ es el conjunto vacío, entonces, tendremos que la gráfica
que obtenemos es aquella donde todos los vértices no son adyacentes entre sí.

Notemos que nuestra definición de gráfica nos permite que un arista tenga como
extremos vértices iguales, aquellas aristas las llamaremos \indice{lazos}, y
gráficamente las representaremos por medio de una linea curva que termina y
empieza en el mismo vértice tal como se muestra en la \cref{fig:02}, aquí los
vértices 1 y 2 ambos tienen lazos.

\begin{figure}[H]
  \centering
  \includegraphics[width=0.25\textwidth]{recursos/capturas/02.jpg}
  \caption{Gráfica de orden dos y tamaño tres, la cual tiene dos lazos.}
  \label{fig:02}
\end{figure}

Una gráfica en la cual todos los vértices tienen lazos, la llamaremos
\indiceSub{gráfica}{reflexiva}. Así la gráfica de \cref{fig:02} es una gráfica
reflexiva. Por el momento trabajaremos exclusivamente con gráficas no
reflexivas, a menos de que se indique lo contrario.

Veamos otro ejemplo, en \cref{fig:02} vimos que había un vértice, el cual no era
adyacente a ningún otro vértice. Supongamos ahora que $V(G)$ es un conjunto
cualquiera con cinco elementos, y que $E(G)$ coincide con el conjunto $\{
\{x,y\} | x,y\in V(G) \}$. Decimos que una gráfica $G$ es una
\textbf{gráfica}\indiceSub{gráfica}{ completa} si y solo si, para cada par de
vértices, se tiene que estos son adyacentes. Repasando nuestras definiciones, en
una gráfica completa no hay ningún par de vértices independientes, mas aún
ningún subconjunto de $V(G)$ es independiente. La gráfica resultante la podemos
representar como se muestra en la \cref{fig:04}.

De forma dual a que un conjunto $S$ sea independiente, tenemos la siguiente
definición, decimos que un conjunto $S\subseteq V(G)$ es un \indice{clan}, si y
si solo si para todo $x,y, x\neq y, xy\in E(G)$.   

En la \cref{fig:04} se tiene que todo subconjunto de $V(G)$ es un clan.

\begin{figure}[H]
  \centering
  \includegraphics[width=0.25\textwidth]{recursos/capturas/04.jpg}
  \caption{Gráfica que tiene todas las aristas posibles, excepto lazos.}.
  \label{fig:03}
\end{figure}

Tenemos un par mas de definiciones. Decimos que dos vértices $u$ y $v$ son
\indiceSub{vértice}{incidente}\textbf{s} en la arista $e$, si $e$ es
precisamente la arista $\{ u,v\}$. Así en la \cref{fig:02} los vértices $u$ y
$v$ son incidentes en $a$. Otra definición mas que tenemos es la siguiente,
decimos que dos aristas son \indice\textbf{arista}{adyacente}\textbf{s} si estas
son de la forma $\{ u,v\}$ y $\{ v,w\}$, es decir comparten un vértice en común.
Por ejemplo las aristas $a$ y $b$ son adyacentes. 

Terminaremos nuestra sección con las siguientes definiciones y ejemplos. Dado
$v\in V(G)$ definimos la \indice{vecindad} de $v$, denotada por $N(v)$ como
aquel subconjunto de $V(G)$ que consta de todos los vértices adyacentes a $v$. Y
al cardinal de $N(v)$ le llamamos el \indiceSub{vértice}{grado} del vértice $v$.
Por otro lado definimos la \textbf{vecindad} \indiceSub{vecindad}{cerrada} de
$v$ como la vecindad de $v$ unión $v$, y la denotamos con $N[v]$. Repasemos
estos conjuntos, en \cref{fig:01} la vecindad del vértice 1, $N(1)= \{ 2 \}$
así el vértice 1 tiene grado 1. Por otro lado el vértice 3, tiene grado cero ya
que su vecindad es vacía.

En \cref{fig:03} notemos que todos los vértices tienen grado 4. La gráficas que
cumplen que todos sus vértices tienen el mismo grado son llamadas
\textbf{gráficas} \indiceSub{gráfica}{regular}\textbf{es}, si además el grado de
sus vértices es $k$ las llamamos $k$-regulares. Así el ejemplo de \cref{fig:03}
es una gráfica 4-regular.

\section{Subgráficas}
\label{sec:SbGrfcs}

A similitud de muchas estructuras en matemáticas, las gráficas también tienen su
respectiva subestructura, así es que introducimos una subgr\'afica. $H$ es
\indice{subgráfica} de una gráfica $G$ si sus vértices y aristas pertenecen
también a $V(G)$, es decir $V(H)\subseteq V(G)$ y $E(H) \subseteq E(G) $.
Decimos que un vértice $v$ es adyacente a una subgráfica $H$ si y solo si existe
un vértice $u\in H$ tal que $uv\in E(G)$.

Otro concepto que surge de forma natural es el de la estructura generada por un
subconjunto $S$. Por lo tanto definimos la \textbf{subgráfica}
\indiceSub{subgráfica}{inducida} por $S\subseteq V(G)$, cuyo conjunto de
vértices es exactamente $S$ y las aristas serán todas las aristas $\{u,v\} $ de
$G$ tales que $u,v \in S$. A esta gráfica inducida la denotaremos con $V[S]$, es
natural que esta es una subgráfica de $G$.

Veamos el siguiente ejemplo. En \cref{fig:04}, tenemos una gráfica y un par de
subconjuntos, el subconjunto de vértices coloreados de color verde y el conjunto
de vértices coloreados de naranja. La gráfica inducida por los vértices verdes
es la gráfica que se muestra en el recuadro de en medio, mientras que la
inducida por el conjunto de vértices naranja es la de la derecha.

\begin{figure}[H]
  \centering
  \includegraphics[width=0.7\textwidth]{recursos/capturas/09.jpg}
  \caption{Gráfica G y subgráficas inducidas $V[\{ a,b,d,i,j,h\}], V[\{
  c,e,g\}]$.}.
  \label{fig:04}
\end{figure}

Denotamos por \indice{S(v)} a la gráfica generada $v$ y todos sus vecinos. Es
decir $S(v)=V[N[v]]$.

Veamos el siguiente ejemplo para aterrizar el concepto. En \cref{fig:05} al
centro mostramos el conjunto $S(a)$ y a la derecha se muestra el conjunto
$S(h)$. Es importante notar que a diferencia de la vecindad cerrada de $a$, los
conjuntos $S(a)$ cuentan con estructura de gráfica, mientras que las vecindades
solo son conjuntos de vértices. Aquellos vértices que cumplen que $S(a)$ es una
gráfica completa, son llamados \textbf{vértices}
\indiceSub{vértice}{simplicial}\textbf{es}.

\begin{figure}[H]
  \centering
  \includegraphics[width=0.73\textwidth]{recursos/capturas/10.jpg}
  \caption{Gráfica $G$ y las vecindades cerradas $S(a)$ y $S(h)$}.
  \label{fig:05}
\end{figure}

Otros ejemplos de subgráficas son los que se obtienen al quitar vértices o
aristas de una subgráfica. Sea $G$ una gráfica y $w\in V(G)$ definimos la
gráfica $G-\{w\}=(V(G)-\{w\}, E')$ donde $uv\in E'$ si y solo si $uv\in E(G)$,
$w\notin uv$. Dado $S\subseteq V(G)$ se puede definir inductivamente $G-S$.
Definimos también $G-\{e\}=(V(G),E')$ con $e\in E(G)$, donde $E'=E(G)-\{e\}$.
Nuevamente, dado $S\subseteq E(G)$ se puede definir inductivamente $G-S$.


\section{Isomorfismos de gráficas}
\label{sec:Isomorfismos}
En esta breve sección abordamos un tema sumamente importante, estudiaremos los
isomorfismos de gráficas.

Veamos el siguiente ejemplo a fin de motivar el estudio de las gráficas
isomorfas.

Tomemos $V(G_1)=\{ 1,2,3\}, E(G_1)=\{ \{1,2\} \}$, $V(G_2)=\{2,3,4 \}, E(G_2)=\{
\{2,3\} \}$ y general $V(G_i)=\{ i, i+1,i+2\}, E(G_i)=\{ i,i+1\} \}$. Esta es
una familia infinita numerable de gráficas y todas ellas pueden ser
representadas con la \cref{fig:01}. Dado lo anterior nos gustaría decir que en
esencia la familia anterior mencionada consta de una sola gráfica. En este punto
es donde surge la definición de gráficas isomorfas, es decir si dos gráficas
pueden ser representadas por el mismo dibujo, diremos que son isomorfas. 

Formalmente, decimos que dos gráficas $G$ y $H$ son
\indiceSub{gráfica}{isomorfa}\textbf{s} si y solo si existe una función
biyectiva $\phi \colon V(G) \to V(H)$ la cual preserva adyacencias, es decir
$uv\in E(G)$ si y solo si $\phi(u)\phi(v)\in E(H)$. Una función que cumple todo
lo anterior es llamado \indice{isomorfismo} de gráficas.

Así, todas las gráficas completas de orden $k$ son isomorfas entre sí. Veamos un
pequeño ejemplo de lo anterior, supongamos que $(V(G),E(G))$, $(V(H),E(H))$ son
ambas gráficas completas de orden $k$, $k\in \mathbb{N}$, y para exhibir un
isomorfismo entre estas gráficas solo basta exhibir una biyección entre $V(G)$ y
$V(H)$, el cual sabemos que siempre existe pues los conjuntos son equipotentes.
Supongamos que $\phi: V(G) \longrightarrow V(H)$ es una biyección, quisieramos
ver que $uv\in E(G)$ si y solo si $\phi(u)\phi(v)\in E(H)$, pero esto es una
consecuencia inmediata del hecho de que ambas gráficas son completas. Así
concluimos que todas las gráficas completas de orden $k$, son isomorfas entre
sí. Finalmente podemos decir que salvo isomorfismos existe una única gráfica
completa de orden $n$, luego a la gráfica de orden $n$ la denotaremos por $K_n$.

Veamos un último ejemplo para concluir la presente sección. En la \cref{fig:07}
tenemos un par de representaciones de gráficas, y nosotros afirmamos que dichas
gráficas son isomorfas, para esto exhibimos el morfismo $\phi$ tal que
$\phi(a)=\overline{a}, \phi(b)=\overline{c}, \phi(c)=\overline{e},
\phi(d)=\overline{b}, \phi(e)=\overline{d} $.

\begin{figure}[H]
  \centering
  \includegraphics[width=0.6\textwidth]{recursos/capturas/11.jpg}
  \caption{Gráficas isomorfas con representaciones distintas.}.
  \label{fig:06}
\end{figure}

Así este es un ejemplo de un par de representaciones de una misma gráfica, salvo
isomorfismos.

\section{Caminos y conexidad.}
\label{sec:CCyT, Conexidad}

Dada una gráfica $G$ y un par de vértices $u,v$ es bastante natural preguntarse
cómo puede llegar uno del vértice $u$ al vértice $v$, siguiendo una sucesión de
vértices adyacentes. Aunque una pregunta m\'as elemental es si es posible llegar
del vértice $u$ al vértice $v$ siguiendo vértices adyacentes. O m\'as a\'un, si
siempre es posible llegar de cualquier v\'ertice a cualquier otro siguiendo
v\'ertices adyacentes.

Dada una gráfica $G$, entendemos por un \indice{camino} a una sucesión de
vértices $(u_1,u_2,...,u_n)$ donde cada par de vértices consecutivos son
adyacentes, es decir $u_{i-1}u_i\in E(G)$, para cada $i \in \{2, \dots, n\}$. Un
camino que empieza en el vértice $u$ y termina en el vértice $v$, lo llamamos
$uv$-camino.

Un camino que no repite vértices es una \indice{trayectoria}, y un camino que no
repite aristas, es un \indice{paseo}. De forma análoga una trayectoria o paseo
que empieza en $u$ y termina en $v$, es una $uv$-trayectoria o un $uv$-paseo. La
longitud de un camino $C$ o es el número de aristas que tiene $C$ y lo denotamos
con $long(C)$, se define de la misma forma la longitud de una trayectoria y un
paseo.


Por un ciclo entenderemos a un camino cuyos vértices inicial y final coinciden.

\begin{figure}[H]
  \centering
  \includegraphics[width=0.6\textwidth]{recursos/capturas/12.jpg}
  \caption{En verde un $am$-camino y en naranja una $am$-trayectoria}.
  \label{fig:07}
\end{figure}

En \cref{fig:08} coloreamos de color verde el $am$-camino
$(a,d,f,e,c,f,g,i,h,e,k,m)$ y tenemos de color naranja la $am$-trayectoria
$(a,b,f,i,m)$.

Otro concepto importante es el de camino cerrado y ciclo. Un
\indiceSub{camino}{cerrado} es un camino que cumple que su vértice inicial es
igual a su vértice final. Por otro lado un camino cerrado que no repite vértices
(intermedios) es un \indice{ciclo}. Se define la longitud de un camino cerrado y
de un ciclo como el número de aristas que tienen estos. Al ciclo de longitud
$n$, lo denotamos con $C_n$.

Adicionalmente tenemos la definición de \indice{distancia} que esta ligada a la
definición de trayectoria. Así dada una gráfica $G=(V(G),E(G))$ se define la
distancia entre dos vértices $u$ y $v$ como $d(u,v)=$m\'in$ \{L(C) \colon\ C$ es
$uv$-trayectoria$ \}$. En caso de no existir alguna $uv$-trayectoria decimos que
$d(u,v)=\infty$. 

En la parte superior de \cref{fig:09} se muestra una gráfica $G$ y un ciclo
$(a,b,c,g,a)$ de longitud cuatro, y en la parte inferior de \cref{fig:09} se
muestra un camino cerrado $(c,b,f,c,g,a,b,a)$ de longitud siete.

\begin{figure}[H]
  \centering
  \includegraphics[width=0.6\textwidth]{recursos/capturas/13.jpg}
  \caption{Un cuatro ciclo y un siete camino cerrado.}.
  \label{fig:08}
\end{figure}

Pasemos ahora al otro concepto importante de la presente sección, el de
conexidad. Decimos que dos vértices  $u$ y $v$ están \indice{conectados} si y
solo si existe una $uv$-trayectoria. Si una gráfica $G$ cumple que existe una
$uv$-trayectoria para todo par de vértices $u$ y $v$ de $G$, decimos que es
\indiceSub{gráfica}{conexa}. Decimos que una gr\'afica es es
\indiceSub{gráfica}{inconexa} si no es conexa. Si $H$ es una subgráfica de $G$,
decimos que es una \indice{componente conexa} de $G$ si no hay otra subgráfica
conexa de $G$ que contenga a $H$.

Observemos \cref{fig:10}, aquí podemos ver que los vértices $d,e$ y el arista
$de$ juegan un papel importante en la conexidad de la gráfica $G$, pues al
remover estos vértices y esta arista, se tiene que la gráfica queda partida en
dos, es decir hay vértices de $G-\{d\}, G-\{e\}$ y $G-\{ de\}$ que ya nos pueden
ser conectados. Por ejemplo en ninguna de las tres gráficas anterior mencionadas
existe una $af$-trayectoria, así las gráficas dejan de ser conexas. Decimos que
un vértice $v$ es un \textbf{vértice} \indiceSub{vértice}{de corte} siempre que
$G-\{v\}$ es una gráfica disconexa. Y en general decimos que $S\subseteq V(G)$
es un \textbf{conjunto}\indiceSub{conjunto}{de corte} si $G-\{S\}$ es una
gráfica disconexa. Otro concepto bastante relacionado al de conjunto de corte es
el de conjunto separador. Dada $G$ y $u,v\in V(G)$, decimos que $S$ es un
\textbf{conjunto} $uv$-\indiceSub{conjunto}{separador} si y solo si al
considerar la gráfica $V-S$ los vértices $u$ y $v$ quedan en componentes conexas
distintas. Además si no existe un subconjunto propio de $S$ que sea
$uv$-separador, decimos que $S$ es un conjunto $uv$-separador mínimo. En
\cref{fig:10} el conjunto $\{a,c,d\}$ es un $bf$-separador, sin embargo no es
mínimo, pues por ejemplo el unitario $\{d \} \subseteq \{a,c,d\}$ y $\{d \}$ es
también un $bf$-separador. 

Definimos también un \indice{puente} como un arista $uv$ tal que $G-\{uv\}$ es
una gráfica disconexa. Así en \cref{fig:10} los vértices $e,d$ son vértices de
corte y el arista $ed$ es un puente. En la gráfica $G-\{ d\}$ nos quedan
justamente dos componentes conexas, a saber, $C_1=V[\{a,b,c \}]$ y
$C_2=V[\{g,e,f\}]$, notemos que en efecto son componentes conexas pues no
existen mas subgráficas $H$ de $G$ que contengan propiamente a $C_1$ y $C_2$. 

\begin{figure}[H]
  \centering
  \includegraphics[width=0.8\textwidth]{recursos/capturas/14.jpg}
  \caption{Vértices de corte y puente.}.
  \label{fig:09}
\end{figure}


\section{Familias especiales de gráficas}
\label{sec:Familias especiales de gráficas}

Es usual querer agrupar a objetos matemáticos en función de sus características,
así en este capítulo exhibimos familias de gráficas, las cuales comparten
propiedades en común, por ejemplo podemos agrupar a gráficas en función de sus
incidencias, sus propiedades de conexidad, sus grados, etc. 

Empecemos con una familia de gráficas que ya se introdujo en \cref{sec:Cncpts
bscs}, las gráficas completas de orden $n$, ya vimos que salvo isomorfismos las
gráficas completas de orden $n$ son únicas y las denotamos con $K_n$, sus
propiedades; cualesquiera dos vértices son adyacentes, es conexa, es cíclica, y
cordal.

A continuación introducimos el concepto de gr\'afica bipartita, una gr\'afica
$G=(V(G), E(G))$ una gráfica. Decimos que $(X,Y)$ es una bipartición de $G$ si y
solo si $X,Y \subseteq V(G)$ y $\{X,Y\}$ es una partición de $V(G)$ tal que para
toda arista $ab \in E(G)$, se cumple que $a$ est\'a en $X$ y $b$ en $Y$.
Aquellas gráficas que admiten una bipartici\'on llaman gráficas bipartitas.


Veamos la siguiente gráfica de la \cref{fig:05}. Aquí es muy fácil decir cual
par de conjuntos de vértices formaran la bipartición de $G$. Recordemos que,
adem\'as de ser una partici\'on, una bipartici\'on debe satisfacer que toda
arista $e \in E(G)$ esta tiene un extremo en $X$ y el otro en $Y$. Así, podemos
proponer que $X$ consista solamente de los vértices de la parte superior y que
$Y$ conste de los vértices que quedan en la parte inferior. Notemos que así las
cosas, sea cual sea la arista que tomemos, esta tiene un extremo arriba y el
otro queda abajo. Por lo tanto es una gráfica bipartita.

\begin{figure}[H]
  \centering
  \includegraphics[width=0.25\textwidth]{recursos/capturas/05.jpg}
  \caption{Gráfica bipartita, con $X=\{$ vértices de arriba $\} $ y $Y=\{$
  vértices de abajo $\}$} .
  \label{fig:10}
\end{figure}

La siguiente gráfica que expondremos es bastante interesante, pues será una
gráfica bipartita la cual además cumple que todo vértice de $X$ es adyacente a
todo vértice de $Y$. Esta condición no se cumple en \cref{fig:05}, por ejemplo
los dos vértices de la derecha, de arriba y abajo, no son adyacentes. Aquellas
gráficas que cumplen esta propiedad, son llamadas gráficas bipartitas completas.
Mas aún si $|X| = m, |Y| = n $ le llamamos gráfica bipartita completa, y dado
que resultan \'única salvo isomorfismos estas se denotan por $K_{m,n}$.


\begin{figure}[H]
  \centering
  \includegraphics[width=0.25\textwidth]{recursos/capturas/06.jpg}
  \caption{Gráfica bipartita $K_{2,3}$.}
  \label{fig:11}
\end{figure}

En \cref{fig:07} tenemos otro ejemplo de gráfica bipartita. En este ejemplo,
toda bipartición será cualquier partición $(X,Y)$ de $V(G)$, en donde alguno de
los dos conjuntos tiene cardinal 1. (Al ser $(X,Y)$, partición de $V(G)$, el
conjunto cuya cardinalidad no sea 1, será el resto de la gráfica.)

\begin{figure}[H]
  \centering
  \includegraphics[width=0.25\textwidth]{recursos/capturas/07.jpg}
  \caption{Gráfica bipartita}
  \label{fig:12}
\end{figure}

Se pueden definir adicionalmente a las gráficas multipartitas, que es una
generalización bastante sencilla de las gráficas bipartitas. Decimos que una
gráfica $G$ es una gráfica $k$-partita si y solo si $V(G)$ puede ser
particionado en $k$ conjuntos independientes, $\{W_i\}_{i=1}^n$. De forma que un
arista $\{u,v\}$ es tal que $u\in W_i, v\in W_j$ con $i,j\in \{1, \dots k\},
i\neq j$ 

Veamos que \cref{fig:07} no solo es 2-partita, si no que es $k$-partita, con
$2\leq k \leq 6$. La partición será de la siguiente forma. Sea $\{W_i\}$ $1\leq
i\leq k-1$ una $(k-1)$-partición de $V-v_0$, donde $v_0 $ es el vértice central.
Y tomamos $W_k=\{v_0\}$. Así, se tiene una $k$-partición.

Otro hecho interesante respecto a las gráficas $k$-partitas es que pueden
colorearse los vértices con $k$ colores distintos de forma que no hay dos
vértices del mismo color que sean adyacentes. Veamos esto en \cref{fig:15}.

\begin{figure}[H]
  \centering
  \includegraphics[width=0.8\textwidth]{recursos/capturas/15.jpg}
  \caption{Distintas coloraciones de una gráfica.}
  \label{fig:13}
\end{figure}

Las coloraciones de gráficas son ampliamente estudiadas, sin embargo las
dejaremos de lado por un momento.

\subsection{\'Arboles}
Decimos que una gráfica $G$ es un árbol si $G$ es acíclica y conexa. Esta
definición depende de dos condiciones, si uno prescinde de la condición de
conexidad, se obtiene la definición de bosque, es decir un bosque es una gráfica
acíclica. Luego las componentes conexas de un bosque son justamente árboles.

\begin{figure}[H]
  \centering
  \includegraphics[width=0.5\textwidth]{recursos/capturas/16.jpg}
  \caption{Árbol.}
  \label{fig:14}
\end{figure}

Es relativamente fácil encontrar ejemplos de árboles, y mas aún teniendo varios
de éstos, es más fácil aún dar ejemplos de bosques. Recordemos que los bosques
solo son aquellos que sus componentes conexas son árboles.

Los árboles tienen una gran cantidad de caracterizaciones, el siguiente teorema
ilustra varias de estas.

\begin{teorema}
\label{teo:101}
    Sea $G$ una gráfica las siguientes proposiciones son equivalentes;\\
    i) $G$ es un árbol.\\
    ii) $G$ es conexa y tiene $n-1$ aristas.\\
    iii) $G$ es conexa y toda arista de $G$ es un puente.\\
    iv) Dados $u,v$ un par de vértices independientes de $G$, existe exactamente
    un ciclo en $G\cup \{uv\}$.
\end{teorema}

Tenemos también el concepto de árbol generador. Así dada una gráfica $G$ decimos
que $T$ es un árbol generador si $T$ es una subgráfica de $G$ la cual es un
árbol y $T$ genera a $G$. Una consecuencia inmediata de que una gráfica $G$
cuente con un árbol generador, se traduce a que $G$ es una gráfica conexa.
También se sabe que toda gráfica conexa posee un árbol generador.

\begin{figure}[H]
  \centering
  \includegraphics[width=0.4\textwidth]{recursos/capturas/17.jpg}
  \caption{Árbol generador en naranja.}
  \label{fig:15}
\end{figure}


\subsection{Gr\'aficas cordales}
Las gráficas cordales juegan un papel de gran importancia respecto a la gráficas
de intervalos en \cref{cap:GrafInt} veremos que ser gráfica cordal es una
necesidad para las gráficas que son de intervalos. 

Dada una gráfica $G$, decimos que es \indiceSub{gráfica}{cordal} si todo ciclo
de longitud mayor o igual a cuatro tiene una cuerda, es decir, un arista de $G$
que no pertenece al ciclo pero que si une a dos vértices del ciclo. En
\cref{fig:09} el único $4$-ciclo no tiene cuerdas, luego $G$ es no cordal.

Las gráficas cordales tienen varias propiedades las cuales ayudan a
caracterizarlas. Una de las primeras características que tiene las gráficas
cordales es la siguiente.

\begin{teorema}
\label{teo:103}
    Toda gráfica cordal $G=(V,E)$ tiene un vértice simplicial. Mas aún si $G$ no
    es una gráfica completa entonces esta tiene un par de vértices smpliciales
    no adyacentes.
\end{teorema}

\begin{proof}
    Si $G$ es una gráfica completa entonces cualquier vértices es simplicial.
    Por lo tanto probemos vía inducción sobre el orden de $G$ que; si $G$ tiene
    un par de vértices $a,b$ no adyacentes y es cordal, entonces $G$ tiene un
    par de vértices simpliciales no adyacentes. Supongamos entonces que para
    toda gráfica de orden menor que $G$ se tienen dos vértices simpliciales. Sea
    $S$ un $ab$-separador mínimo y denotemos con $A,B$ a las componentes conexas
    de $G-S$ que contienen a $a$ y $b$ respectivamente. Tenemos que $G[C \cup
    S]$ tiene dos vértices simpliciales para $C\in \{A,B\}$ o bien $G[C \cup S]$
    es una gráfica completa, en el primer caso alguno de los dos vértices
    simpliciales debe pertenecer a $C$ pues $S$ es completa como consecuencia
    del teorema anterior. En el segundo caso, todo vértice de $S$ es simplicial
    en $G[C \cup S]$. Dado que $V[C]\subseteq G[C \cup S]$ se tiene que todo
    vértice simplicial de $G[C \cup S]$ que esté en $C$ es simplicial en $G$,
    así se tiene que hay dos vértices simpliciales.
\end{proof}

\begin{teorema}
    Cada gráfica finita cordal contiene un vértice simplicial.
\end{teorema}

\begin{proof}
    Prueba por inducción sobre el orden de $G$.\\
    Claramente si el orden de $G$ es uno, hay un punto simplicial, a saber el
    único vértice de $G$.
    
    Supongamos entonces que toda gráfica finita acíclica de orden menor a $n$,
    $n\in \mathbb{N}$, satisface tener un vértice simplicial.

    Veamos ahora que una gráfica $G$ finita acíclica de orden $n$ tiene un punto
    simpicial. Sea $b\in V(G)$ un vértice arbitrario, y a continuación
    consideremos $G_1=G-\{b\}$, por hipótesis inductiva sabemos que existe un
    $a\in G_1$ que es un vértice simplicial, llamemos $S_1(a)=[S(a)]_{G_1}$. En
    este punto notemos los siguientes tres casos; \\
    i) Si $a, b$ no son adyacentes, entonces tenemos que $a$ es también
    simplicial en $G$. Y tenemos el resultado que buscábamos.\\
    ii) Si existe algún $c\in S_1(a)$ no tiene vecinos en $G-S_1(a)$, entonces
    $S(c)=S_1(a)$, que sabemos que es una gráfica completa, y por consecuencia
    $c$ es vértice simplicial de $G$.\\
    iii) Como último caso trivial se tiene que si $b$ es adyacente a todos los
    puntos de $S_1(a)$ entonces $S(a)=S_1(a) \cup \{b\}$ y entonces $a$ es
    simplicial. \\
    Por todo lo anterior nos centraremos a considerar el caso en el que;\\
    I) $ab \in E(G)$ \\
    II) $\forall c\in S_1(a), c\neq a, (N(c)\nsubseteq S_1(a))$\\
    III) $\exists c_0 \in S_1(a), c\neq a,(c_0b\notin E(G))$\\
    Consideremos $G-S_1(a)$, esta gráfica no es conexa necesariamente, por lo
    cual, llamamos $C_1$ a la componente conexa de $G-\{S_1(a)\}$ que contiene a
    $b$. Llamamos $C_2$ a $G-\{C_1 \cup S_1(a)\}$. \\
    A continuación probaremos que si $c\in S_1(a)$ y $c$ es vecino a la
    componente $C_1$, entonces $cb\in E(G)$. Así las cosas sea $c\in S_1(a)$,
    tal que $c$ es vecino de la subgráfica $C_1$, y tomemos $d_1\in C_1$ un
    vértice adyacente a $c$. Notemos que si $c=a$ o $d_1=b$ entonces es
    inmediato que $cb\in E(G)$. Por lo que supondremos que $c \neq a$ y $d_1\neq
    b$. Como primer punto, tenemos que dado que $C_1$ es una componente conexa,
    y $d_1,b\in C_1$ entonces existe una trayectoria $d_1,\dots,d_k,b$ en $C_1$
    con $b\neq d_i$ para todo $i\in \{1,\dots , k\}$. Por otro lado dado que
    $a,b$ son adyacentes, se tiene que $b,a,c,d_1,\dots,d_k,b$ es un ciclo.
    Además se tiene que $a\neq d_i$ y $ad_i\notin E(G)$ para todo $i\in
    \{1,\dots , k\}$.
\end{proof}

La primer caracterización que exhibimos, nos habla sobre dar un orden de los
vértices de $G$.

Sea $G$ una gráfica y sea $\sigma= [v_1,v_2, \dots, v_n]$ una ordenación de los
vértices de $G$. Decimos que $\sigma$ es un \indice{esquema perfecto de
eliminación de vértices} si y solo si el vértice $v_i$ es un vértice simplicial
de la gráfica inducida $V[v_{i}, \dots , v_n]$.

Un ejemplo de una gráfica con un esquema perfecto de eliminación de vértices lo
mostramos en \cref{fig:16}, el esquema $\sigma=[1,2,\dots, 10]$ es un esquema
perfecto de eliminación vértices. Veamos que; el vértice $1$ es un vértice
simplicial en $G$, el vértice $2$ es un vértice simplicial en $G[2, \dots, 10]$,
análogamente el $3$ es un vértice simplicial en $G[3, \dots, 10]$ y así
consecutivamente. 

Notemos que una gráfica puede tener mas de un esquema perfecto de eliminación de
vértices. En el ejemplo anterior citado $\phi = [1,3,2,4,5,6,7,8,9,10]$ es otro
esquema perfecto de eliminación de vértices. En realidad siempre podemos empezar
un esquema perfecto de eliminación de vértices a partir de un vértice simplicial
(en la gráfica inducida por los vértices restantes).

\begin{figure}[H]
  \centering
  \includegraphics[width=0.75\textwidth]{recursos/capturas/19.jpg}
  \caption{Esquema perfecto de eliminación de vértices. En naranja se muestran los $S(i)$.}
  \label{fig:16}
\end{figure}

Así tenemos el siguiente teorema. 

\begin{teorema}
\label{teo:102}
    Sea $G$ una gráfica. Las siguientes proposiciones son equivalentes.\\
    (i) $G$ es una gráfica cordal.\\
    (ii) $G$ tiene un esquema perfecto de eliminación de vértices.\\
    (iii) Cada vértice mínimo separador de vértices induce una subgráfica
    completa de $G$.
\end{teorema}

\begin{proof}
    (iii) $\Rightarrow$ (i) Sea $[a,x,b, y_1, \dots, y_n, a]$ un ciclo simple de
    $G$. Cualquier $ab$-separador debe contener a los vértices $x,y_i$ para
    alguna $i\in \{1, \dots, n \}$, luego al pertenecer a un $ab$-separador y al
    tener por hipótesis que los separadores inducen subgráficas completas, se
    tiene que $xy_i \in E(G)$, luego $[a,x,b, y_1, \dots, y_n, a]$ tiene una
    cuerda.\\
    
    (i) $\Rightarrow$ (iii) Supongamos que $S$ es un $ab$-separador mínimo y
    denotemos con $A,B$ a las componentes conexas de $G-S$ que contienen a $a$ y
    $b$ respectivamente. Dado que $S$ es minimal debe suceder todo elemento
    $x\in S$ es vecino de $A$ y de $B$. Por lo tanto para cada par de vértices
    $x,y\in S$ existen los caminos $[x,a_1, \dots, a_r,y]$ y $[y,b_1,\dots,
    b_t,x]$ de tal forma que $a_i\in A, b_i\in B$ y además son elegidos de forma
    que las longitudes de estos es mínima. Luego al concatenar los caminos
    anteriores, obtenemos el ciclo $[x,a_1, \dots, a_r,y,b_1,\dots, b_t,x]$ el
    cual es un ciclo de longitud al menos 4, luego por hipótesis tenemos que
    debe haber una cuerda. De aquí notamos que $a_ib_j\notin E$ pues $A$ y $B$
    están en componentes distintas. También $a_ia_j, b_ib_j \notin E$ para $i,j$
    no consecutivos, dada la minimalidad de las trayectorias. Por lo tanto la
    unica cuerda posible es $xy$.
    
\end{proof}

Para concluir la prueba \cref{teo:202} probaremos un teorema auxiliar. Una vez
teniendo este teorema, regresamos a la prueba del \cref{teo:202}. 
\begin{proof}
    (i) $\Rightarrow$ (ii) Probemos esto por inducción. Supongamos que toda
    gráfica cordal de grado menor al de $G$ tiene un esquema perfecto de
    eliminación de vértices. Probemos que $G$ en efecto tiene un esquema
    perfecto de eliminación de vértices. Usando \cref{teo:203}, tenemos que $G$
    tiene un vértice simplicial $x$. Al considerar $G-\{x\}$ tenemos una gráfica
    cordal luego por inducción $G-\{x\}$ cuenta con un esquema perfecto de
    eliminación de vértices $\sigma $. Para obtener un esquema perfecto de
    eliminación de vértices de $G$, basta adjuntar a $\sigma$ el vértice $x$
    como prefijo. (ii) $\Rightarrow$ (i) 
    
\end{proof}

Tenemos la propiedad probada por Dirac, la cual establece que toda gráfica
cordal tiene al menos un vértice simplicial. En cuanto a las caracterizaciones
de las gráficas cordales, tenemos que toda gráfica cordal tiene un esquema
perfecto de eliminación de vértices, dicho esquema lo definimos mas adelante.
Otra caracterización mas es el hecho de que todo conjunto minimo separador de
vértices induce una gráfica completa y 



\begin{comment}
En la \cref{fig:17} mostramos una gráfica la cual $\sigma = [a,e,b,f,c,g,d]$ es
un esquema perfecto de eliminación de vértices. Notemos que una gráfica puede
tener mas de un esquema perfecto de eliminacióin de vértices, por ejemplo en la
anterior citada $\phi = [d,g,c,f,b,e,a]$ es otro esquema perfecto de eliminación
de vértices.

\begin{figure}[H]
  \centering
  \includegraphics[width=0.5\textwidth]{recursos/capturas/18.jpg}
  \caption{Árbol generador en naranja.}
  \label{fig:17}
\end{figure}  
\end{comment}


\begin{comment}
\section{C\'omo usar esta plantilla}
\label{sec:howto}

Esta plantilla se dise\~n\'o como una ayuda para aquellos usuarios que ya
est\'an familiarizados con \LaTeX, pero nunca han desarrollado un proyecto
``grande'' (m\'as all\'a de tareas o reportes finales de proyectos).   Siguiendo
las instrucciones encontradas en el archivo \ttt{README.md}, lo m\'as probable
es que hayan creado un nuevo repositorio a partir del ``template repository''
que contiene este proyecto.   En primer lugar, verifique que el proyecto compile
adecuadamente; el proyecto deber\'ia de compilar sin errores ni advertencias. Es
posible que la primera vez que se compila, su manejador de paquetes actualice
varios de \'estos, lo que puede llevar un tiempo.   En caso de tener errores, es
posible que \'estos se deban a la falta de algunos paquetes, y a que su
manejador de paquetes no los instala autom\'aticamente;  instalar los paquetes
faltantes manualmente deber\'ia de resolver todos los problemas.

La estructura de este proyecto es sencilla.   Hay un archivo central,
\ttt{tesis.tex}, que contiene el pre\'ambulo del documento, y donde se incluyen
todos los paquetes y definiciones necesarias.   El c\'odigo est\'a comentado,
explicando de forma m\'inima para qu\'e sirve cada comando; se recomienda que al
modificarlo se mantenga un estilo semejante para no causarle problemas
innecesarios a su yo del futuro.   Todos los contenidos se encuentran en otros
archivos dentro del mismo directorio, que son llamados desde \ttt{tesis.tex}
mediante el comando \ttt{\textbackslash{include}}.   De esta forma se incluyen
la car\'atula, la hoja de datos, los cap\'itulos que forman parte de la tesis,
la bibliograf\'ia, etc.   Por otro lado, \LaTeX~ genera (casi) autom\'aticamente
el \'indice y el \'indice alfab\'etico, pero hay que agregar comandos para su
inclusi\'on.  La mayor\'ia de los usuarios s\'olo necesitan preocuparse por
modificar algunos de los archivos existentes, e incluir otros. Sin embargo, es
\'util que est\'en familiarizados con los conceptos de \ttt{frontmatter},
\ttt{mainmatter}, \ttt{appendix} y \ttt{backmatter} (puede referirse a
\cite{oetiker2007} para revisarlos).

A continuaci\'on, se recomienda revisar el documento generado (este documento) e
identificar cu\'ales son las caracter\'isticas que se desean utilizar (dibujos,
algoritmos, tablas, etc.).   Tras determinar cu\'ales son los paquetes
relevantes para las caracter\'isticas deseadas, comentar (o borrar) todos
aquellos que no ser\'an utilizados en el archivo \ttt{tesis.tex}.   Si se est\'a
usando \ttt{git}, se recomienda leer \cref{sec:git}.   De otro modo, puede
empezar a reemplazar los contenidos de la plantilla con su propio trabajo.

\section[Uso recomendado con git]{Flujo de trabajo recomendado con \ttt{git}}
\label{sec:git}

Si el lector no est\'a usando \ttt{git}\index{git}, puede ignorar esta
secci\'on.   De otro modo, se propone un flujo de trabajo con el que el tesista
puede autogestionar el desarrollo de su tesis, o \'este puede ser supervisado
por su director de tesis mediante el uso de \ttt{GitHub}\index{git!GitHub}.

Este repositorio cuenta con dos ramas al momento de ser clonado: \ttt{master} y
\ttt{original}.   Idealmente, \ttt{master} debe contener su trabajo final, una
vez que ha sido revisado por su director de tesis, por lo que nunca deber\'ia de
trabajar directamente sobre esta rama.   Por este motivo, antes de realizar
cambios y experimentos en los archivos del proyecto, se recomienda crear una
nueva rama, llamada por ejemplo \ttt{prueba}, usando el comando \ttt{git
checkout -b prueba}.   Tras realizar algunos experimentos, eliminar los
contenidos que no necesita, y agregar sus datos a la car\'atula y hoja de datos,
posiblemente se sienta listo para empezar a incluir su trabajo en el proyecto.
En este momento se recomienda agregar los cambios realizados al repositorio,
realizar un \ttt{commit} con los mismos, y realizar un \ttt{merge} a
\ttt{master}.   A partir de ahora, \ttt{master} estar\'a lista para empezar a
trabajar.

En este momento, es posible crear
\href{https://guides.github.com/features/issues/}{\ttt{Issues}} en su
repositorio para tener metas de trabajo.   Como muy posiblemente s\'olo una
persona est\'e trabajando en el proyecto (el tesista), es posible que s\'olo se
trabaje en un \ttt{issue} a la vez, sin embargo, es una buena pr\'actica tener
una rama para cada \ttt{issue} (lo que resultar\'a a\'un m\'as \'util si se
trabaja en m\'as de una caracter\'istica nueva a la vez).   Idealmente, toda
rama nueva saldr\'a de \ttt{master}, y estar\'a dedicada a resolver un \'unico
\ttt{issue}.   Un ciclo de trabajo\index{ciclo de trabajo} usual puede verse de
la siguiente forma.

\begin{enumerate}
  \item Determinar una caracter\'istica nueva que se desea agregar al trabajo
    (e.g., la demostraci\'on de un teorema central de la tesis).

  \item Crear un \ttt{issue} describiendo qu\'e es lo que espera agregar al
    trabajo (e.g., qu\'e conceptos se necesitan agregar, proveer una referencia
    del teorema, indicar si es necesario incluir resultados preliminares o
    ejemplos).

  \item Asignar el \ttt{issue} al tesista, y opcionalmente agregar una fecha
    l\'imite.   (En caso de que el director de tesis est\'e supervisando el
    trabajo mediante \ttt{git}, asignarlo como revisor del \ttt{issue}.)

  \item Crear una nueva rama (a partir de \ttt{master}) para resolver el
    \ttt{issue}.

  \item Una vez resuelto el \ttt{issue}, hacer un \ttt{commit} (o varios) con
    los cambios, un \ttt{push} al repositorio, y abrir un \ttt{pull request} que
    ser\'a cerrado una vez que el director de tesis haya revisado el nuevo
    trabajo. (En caso de que el director de tesis est\'e supervisando el trabajo
    mediante \ttt{git}, deber\'a de ser agregado como revisor del \ttt{pull
    request}, y \'este ser\'a mezclado hasta tener su aprobaci\'on.)

  \item Tras aceptar el \ttt{pull request}, cerrar el \ttt{issue} y borrar la
    rama correspondientes (esto puede hacerse autom\'aticamente al aceptar el
    \ttt{pull request}).
\end{enumerate}

Un ciclo de trabajo tomar\'a tipicamente una semana, por lo que las metas a ser
cubiertas por cada \ttt{issue} deber\'an planearse con cuidado.

Se recomienda no modificar la rama \ttt{original}.   Si en cualquier momento se
necesitara tener acceso a este documento (quiz\'a el usuario requiere revisar un
ejemplo, o recuperar alg\'un paquete que borr\'o previamente), basta con
cambiarse a la rama \ttt{original}, donde siempre habr\'a una copia local del
mismo.   Es importante se\~nalar que, por el momento, \ttt{GitHub} crea
historias distintas para todas las ramas en un repositorio plantilla, por lo que
no es posible mezclar f\'acilmente commits entre \ttt{original} y \ttt{master}.
Esperamos que en un futuro \ttt{GitHub} permita empezar todas las ramas en un
repositorio plantilla con el mismo commit, en cuyo caso se integrar\'a una
tercera rama (\ttt{prueba}) a este repositorio.
\end{comment}