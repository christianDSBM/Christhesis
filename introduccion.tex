\chapter{Introducci\'on}
\label{sec:intro}

Este documento tiene algunos ejemplos m\'inimos de caracter\'isticas que se
suelen utilizar en tesis de las licenciaturas en matem\'aticas y ciencias de la
computaci\'on, en particular en el \'area de teor\'ia de gr\'aficas (el \'area
de trabajo del autor).

Se exhorta al usuario a leer la
\href{https://tobi.oetiker.ch/lshort/lshort.pdf}{Not So Short Inroduction to
\LaTeX}.   Aunque realmente no es un documento muy largo, para quienes nunca han
usado \LaTeX{} es posible que las partes t\'ecnicas no tengan sentido.   En este
caso, es recomendable leer los dos primeros cap\'itulos, y regresar al resto del
documento para hacer consultas, o cuando se tenga algo de experiencia y se
desee mejorar como usuario.   En particular, antes de intentar cambiar el tipo
de letra, o el tama\~no de los m\'argenes, considere la siguiente observaci\'on
que aparece en el documento antes mencionado:
\begin{quote}
  Typographical design  is  a  craft.   Unskilled  authors  often  commit
  seriousformatting errors  by  assuming  that  book  design  is  mostly  a
  question of aesthetics---``If a document looks good artistically, it is well
  designed.'' But as a document has to be read and not hung up in a picture
  gallery, the readability and understandability is of much greater importance
  than the beautiful look of it.
\end{quote}

Idealmente, el lector obtuvo esta plantilla mediante
\href{https://github.com/Japodrilo/template-tesis}{este repositorio}. De no ser
as\'i, se le invita a visitarlo, y a usar las bondades del control de versiones
que el uso de \href{https://git-scm.com/}{Git} otorga (en particular cuando se
utiliza en conjunto con alguna plataforma para albergar sus repositorios
remotamente\footnote{Los alumnos de la Facultad de Ciencias de la UNAM tienen
acceso al \href{https://education.github.com/pack}{GitHub Student Developer
Pack} con su cuenta \ttt{@ciencias.unam.mx}.}).


\section{C\'omo usar esta plantilla}
\label{sec:howto}

Esta plantilla se dise\~n\'o como una ayuda para aquellos usuarios que ya
est\'an familiarizados con \LaTeX, pero nunca han desarrollado un proyecto
``grande'' (m\'as all\'a de tareas o reportes finales de proyectos).   Siguiendo
las instrucciones encontradas en el archivo \ttt{README.md}, lo m\'as probable
es que hayan creado un nuevo repositorio a partir del ``template repository''
que contiene este proyecto.   En primer lugar, verifique que el proyecto compile
adecuadamente; el proyecto deber\'ia de compilar sin errores ni advertencias. Es
posible que la primera vez que se compila, su manejador de paquetes actualice
varios de \'estos, lo que puede llevar un tiempo.   En caso de tener errores, es
posible que \'estos se deban a la falta de algunos paquetes, y a que su
manejador de paquetes no los instala autom\'aticamente;  instalar los paquetes
faltantes manualmente deber\'ia de resolver todos los problemas.

La estructura de este proyecto es sencilla.   Hay un archivo central,
\ttt{tesis.tex}, que contiene el pre\'ambulo del documento, y donde se incluyen
todos los paquetes y definiciones necesarias.   El c\'odigo est\'a comentado,
explicando de forma m\'inima para qu\'e sirve cada comando; se recomienda que al
modificarlo se mantenga un estilo semejante para no causarle problemas
innecesarios a su yo del futuro.   Todos los contenidos se encuentran en otros
archivos dentro del mismo directorio, que son llamados desde \ttt{tesis.tex}
mediante el comando \ttt{\textbackslash{include}}.   De esta forma se
incluyen la car\'atula, la hoja de datos, los cap\'itulos que forman parte de la
tesis, la bibliograf\'ia, etc.   Por otro lado, \LaTeX~ genera (casi)
autom\'aticamente el \'indice y el \'indice alfab\'etico, pero hay que agregar
comandos para su inclusi\'on.  La mayor\'ia de los usuarios s\'olo necesitan
preocuparse por modificar algunos de los archivos existentes, e incluir otros.
Sin embargo, es \'util que est\'en familiarizados con los conceptos de
\ttt{frontmatter}, \ttt{mainmatter}, \ttt{appendix} y \ttt{backmatter} (puede
referirse a \cite{oetiker2007} para revisarlos).

A continuaci\'on, se recomienda revisar el documento generado (este documento) e
identificar cu\'ales son las caracter\'isticas que se desean utilizar (dibujos,
algoritmos, tablas, etc.).   Tras determinar cu\'ales son los paquetes
relevantes para las caracter\'isticas deseadas, comentar (o borrar) todos
aquellos que no ser\'an utilizados en el archivo \ttt{tesis.tex}.   Si se est\'a
usando \ttt{git}, se recomienda leer \cref{sec:git}.   De otro modo, puede
empezar a reemplazar los contenidos de la plantilla con su propio trabajo.

\section[Uso recomendado con git]{Flujo de trabajo recomendado con \ttt{git}}
\label{sec:git}

Si el lector no est\'a usando \ttt{git}\index{git}, puede ignorar esta
secci\'on.   De otro modo, se propone un flujo de trabajo con el que el tesista
puede autogestionar el desarrollo de su tesis, o \'este puede ser supervisado
por su director de tesis mediante el uso de \ttt{GitHub}\index{git!GitHub}.

Este repositorio cuenta con dos ramas al momento de ser clonado: \ttt{master} y
\ttt{original}.   Idealmente, \ttt{master} debe contener su trabajo final, una
vez que ha sido revisado por su director de tesis, por lo que nunca deber\'ia de
trabajar directamente sobre esta rama.   Por este motivo, antes de realizar
cambios y experimentos en los archivos del proyecto, se recomienda crear una
nueva rama, llamada por ejemplo \ttt{prueba}, usando el comando \ttt{git
checkout -b prueba}.   Tras realizar algunos experimentos, eliminar los
contenidos que no necesita, y agregar sus datos a la car\'atula y hoja de datos,
posiblemente se sienta listo para empezar a incluir su trabajo en el proyecto.
En este momento se recomienda agregar los cambios realizados al repositorio,
realizar un \ttt{commit} con los mismos, y realizar un \ttt{merge} a
\ttt{master}.   A partir de ahora, \ttt{master} estar\'a lista para empezar a
trabajar.

En este momento, es posible crear
\href{https://guides.github.com/features/issues/}{\ttt{Issues}} en su
repositorio para tener metas de trabajo.   Como muy posiblemente s\'olo una
persona est\'e trabajando en el proyecto (el tesista), es posible que s\'olo
se trabaje en un \ttt{issue} a la vez, sin embargo, es una buena pr\'actica
tener una rama para cada \ttt{issue} (lo que resultar\'a a\'un m\'as \'util si
se trabaja en m\'as de una caracter\'istica nueva a la vez).   Idealmente, toda
rama nueva saldr\'a de \ttt{master}, y estar\'a dedicada a resolver un \'unico
\ttt{issue}.   Un ciclo de trabajo\index{ciclo de trabajo} usual puede verse
de la siguiente forma.

\begin{enumerate}
  \item Determinar una caracter\'istica nueva que se desea agregar al
    trabajo (e.g., la demostraci\'on de un teorema central de la tesis).

  \item Crear un \ttt{issue} describiendo qu\'e es lo que espera agregar
    al trabajo (e.g., qu\'e conceptos se necesitan agregar, proveer una
    referencia del teorema, indicar si es necesario incluir resultados
    preliminares o ejemplos).

  \item Asignar el \ttt{issue} al tesista, y opcionalmente agregar una fecha
    l\'imite.   (En caso de que el director de tesis est\'e supervisando el
    trabajo mediante \ttt{git}, asignarlo como revisor del \ttt{issue}.)

  \item Crear una nueva rama (a partir de \ttt{master}) para resolver el
    \ttt{issue}.

  \item Una vez resuelto el \ttt{issue}, hacer un \ttt{commit} (o varios) con
    los cambios, un \ttt{push} al repositorio, y abrir un \ttt{pull request}
    que ser\'a cerrado una vez que el director de tesis haya revisado el nuevo
    trabajo. (En caso de que el director de tesis est\'e supervisando el
    trabajo mediante \ttt{git}, deber\'a de ser agregado como revisor del
    \ttt{pull request}, y \'este ser\'a mezclado hasta tener su aprobaci\'on.)

  \item Tras aceptar el \ttt{pull request}, cerrar el \ttt{issue} y borrar la
    rama correspondientes (esto puede hacerse autom\'aticamente al aceptar el
    \ttt{pull request}).
\end{enumerate}

Un ciclo de trabajo tomar\'a tipicamente una semana, por lo que las metas a ser
cubiertas por cada \ttt{issue} deber\'an planearse con cuidado.

Se recomienda no modificar la rama \ttt{original}.   Si en cualquier momento se
necesitara tener acceso a este documento (quiz\'a el usuario requiere revisar un
ejemplo, o recuperar alg\'un paquete que borr\'o previamente), basta con
cambiarse a la rama \ttt{original}, donde siempre habr\'a una copia local del
mismo.   Es importante se\~nalar que, por el momento, \ttt{GitHub} crea
historias distintas para todas las ramas en un repositorio plantilla, por lo que
no es posible mezclar f\'acilmente commits entre \ttt{original} y \ttt{master}.
Esperamos que en un futuro \ttt{GitHub} permita empezar todas las ramas en un
repositorio plantilla con el mismo commit, en cuyo caso se integrar\'a una
tercera rama (\ttt{prueba}) a este repositorio.
